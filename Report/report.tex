\documentclass[conference]{IEEEtran}
\usepackage{mathptmx} % Times New Roman font
\usepackage{graphicx}
\usepackage{array}
\usepackage{cite}
\usepackage{amsmath}
\usepackage{balance} % For balancing last page columns

% Set margins for US Letter paper
\usepackage[letterpaper,
            left=0.625in,
            right=0.625in,
            top=0.75in,
            bottom=1in]{geometry}

% Section numbering with Roman numerals
\renewcommand{\thesection}{\Roman{section}}
\renewcommand{\thesubsection}{\Alph{subsection}}

\begin{document}

\title{Implementation and In-Depth Analysis of Flight Optimization Systems for Travel Agencies: A Multi-Criteria Approach}

\author{Yasin Yeşilyurt\\
TOBB ETÜ Artificial Intelligence
Engineering\\
Söğütözü caddesi TOBB ETÜ Konukevi\\
Yenimahalle/Ankara\\
yasinyesilyurt@hotmail.com}

\maketitle

% \begin{abstract}
% This paper addresses the challenge of finding the best possible flight route for passengers.
% The problem is formulated as a graph pathfinding problem. And algorithms, A*, Dijkstra's and Genetic algorithm are compared on their optimality, time and effectiveness of finding the optimal path. 
% The algorithm is implemented in Python and tested on a dataset of past flight routes on United States. 
% The results show that the A* algorithm is effective in finding the optimal flight route, with a significant reduction in travel time compared to traditional methods.
% \end{abstract}

\begin{abstract}
This paper presents a multi-criteria flight optimization system for travel agencies, designed to recommend optimal flight routes by balancing cost, duration, and customer satisfaction. 
The proposed framework employs A, Dijkstra's, and Genetic Algorithms implemented in Python to evaluate flight data spanning 1993-2023 from U.S. domestic routes, sourced from a Kaggle dataset containing 245,955 entries with parameters such as fare, distance, carrier dominance, and airport coordinates. 
Key innovations include a hybrid approach combining heuristic pathfinding (using Haversine distance for A*) with genetic crossover operations to -> !!!!!! <- address multi-objective optimization. 
Eesults demonstrate the effectiveness of A* and Dijkstra's algorithms in minimizing travel costs and time, while genetic algorithm gives sufficient results it is much slover than mentioned algorithms without proper optimization techniques. 
The system currently provides personalized flight recommendations based on user preferences, with visualized outputs generated via the NetworkX library.
This work contributes a scalable framework for enhancing decision-making in travel planning systems.
\end{abstract}

\section{INTRODUCTION}
Your goal is to simulate the usual appearance of papers in an \textit{IEEE conference proceedings}. Prepare your paper in full-size format, on US letter paper (8$\frac{1}{2}$ by 11 inches). For A4 paper, use the A4 settings.

\subsection{Type Sizes and Typefaces}
Follow the type sizes specified in Table~\ref{tab:type}. Times New Roman is the preferred font.

\begin{table}[ht]
\centering
\caption{Type Sizes for Papers}
\label{tab:type}
\begin{tabular}{|l|l|l|l|}
\hline
\textbf{Type size (pts.)} & \textbf{Regular} & \textbf{Bold} & \textbf{Italic} \\ \hline
6 & Table captions & & \\ \hline
8 & Section titles & & \\ \hline
9 & Main text & Abstract & Subheading \\ \hline
10 & Authors' names & & \\ \hline
11 & Paper title & & \\ \hline
\end{tabular}
\end{table}

\section{HELPFUL HINTS}
\subsection{Figures and Tables}
Position figures and tables at the tops and bottoms of columns. Figure captions should be centered below the figures as shown in Fig.~\ref{fig:sample}.

\begin{figure}[ht]
\centering
% \includegraphics[width=3in]{sample-figure}
\caption{Sample figure caption.}
\label{fig:sample}
\end{figure}

\subsection{References}
Number citations consecutively in square brackets~\cite{ref1}. Use ``Ref.~[3]'' at the beginning of a sentence. 

\subsection{Equations}
Number equations consecutively:
\begin{equation}
a + b = c
\end{equation}
Symbols should be defined immediately following the equation.

\section{UNITS}
Use either SI (MKS) or CGS as primary units. Avoid combining SI and CGS units.

\section{SOME COMMON MISTAKES}
The word ``data'' is plural. Use proper punctuation within quotation marks: ``like this.'' 

\section*{ACKNOWLEDGMENT}
The preferred spelling is ``acknowledgment'' without an ``e'' after the ``g.''

\balance % Balance columns on last page

\begin{thebibliography}{9}
\bibitem{ref1} G. Eason et al., ``On certain integrals,'' Phil. Trans. Roy. Soc. London, vol. A247, pp. 529-551, 1955.
\bibitem{ref2} J. Clerk Maxwell, A Treatise on Electricity and Magnetism. Oxford: Clarendon, 1892.
\end{thebibliography}

\end{document}