\documentclass[conference]{IEEEtran}
\usepackage{mathptmx} % Times New Roman font
\usepackage{graphicx}
\usepackage{array}
\usepackage{cite}
\usepackage{amsmath}
\usepackage{balance} % For balancing last page columns

% Set margins for US Letter paper
\usepackage[letterpaper,
            left=0.625in,
            right=0.625in,
            top=0.75in,
            bottom=1in]{geometry}

% Section numbering with Roman numerals
\renewcommand{\thesection}{\Roman{section}}
\renewcommand{\thesubsection}{\Alph{subsection}}

\begin{document}

\title{Preparation of Papers in Two-Column Format for IEEE Conference Proceedings}

\author{J.~Q.~Author\\
IEEE Conference Publishing\\
445 Hoes Lane\\
Piscataway, NJ 08854 USA}

\maketitle

\begin{abstract}
These instructions give you basic guidelines for preparing papers for conference proceedings. Follow the type sizes and formatting specifications carefully to ensure proper appearance.
\end{abstract}

\section{INTRODUCTION}
Your goal is to simulate the usual appearance of papers in an \textit{IEEE conference proceedings}. Prepare your paper in full-size format, on US letter paper (8$\frac{1}{2}$ by 11 inches). For A4 paper, use the A4 settings.

\subsection{Type Sizes and Typefaces}
Follow the type sizes specified in Table~\ref{tab:type}. Times New Roman is the preferred font.

\begin{table}[ht]
\centering
\caption{Type Sizes for Papers}
\label{tab:type}
\begin{tabular}{|l|l|l|l|}
\hline
\textbf{Type size (pts.)} & \textbf{Regular} & \textbf{Bold} & \textbf{Italic} \\ \hline
6 & Table captions & & \\ \hline
8 & Section titles & & \\ \hline
9 & Main text & Abstract & Subheading \\ \hline
10 & Authors' names & & \\ \hline
11 & Paper title & & \\ \hline
\end{tabular}
\end{table}

\section{HELPFUL HINTS}
\subsection{Figures and Tables}
Position figures and tables at the tops and bottoms of columns. Figure captions should be centered below the figures as shown in Fig.~\ref{fig:sample}.

\begin{figure}[ht]
\centering
% \includegraphics[width=3in]{sample-figure}
\caption{Sample figure caption.}
\label{fig:sample}
\end{figure}

\subsection{References}
Number citations consecutively in square brackets~\cite{ref1}. Use ``Ref.~[3]'' at the beginning of a sentence. 

\subsection{Equations}
Number equations consecutively:
\begin{equation}
a + b = c
\end{equation}
Symbols should be defined immediately following the equation.

\section{UNITS}
Use either SI (MKS) or CGS as primary units. Avoid combining SI and CGS units.

\section{SOME COMMON MISTAKES}
The word ``data'' is plural. Use proper punctuation within quotation marks: ``like this.'' 

\section*{ACKNOWLEDGMENT}
The preferred spelling is ``acknowledgment'' without an ``e'' after the ``g.''

\balance % Balance columns on last page

\begin{thebibliography}{9}
\bibitem{ref1} G. Eason et al., ``On certain integrals,'' Phil. Trans. Roy. Soc. London, vol. A247, pp. 529-551, 1955.
\bibitem{ref2} J. Clerk Maxwell, A Treatise on Electricity and Magnetism. Oxford: Clarendon, 1892.
\end{thebibliography}

\end{document}